\documentclass{article}
\usepackage{amsmath}
\usepackage{graphicx}
\usepackage{hyperref}
\usepackage{geometry}

% Set smaller margins
\geometry{margin=1in}

\title{SemEval-2025 Task-3 — Mu-SHROOM}
\author{}
\date{}

\begin{document}

\maketitle

\section{Introduction}

This task is part of SemEval-2025 Task-3 — Mu-SHROOM, the Multilingual Shared-task on Hallucinations and Related Observable Overgeneration Mistakes, which builds on the previous SHROOM task with key changes: it covers multiple languages including Arabic (Modern Standard), Chinese (Mandarin), English, Finnish, French, German, Hindi, Italian, Spanish, and Swedish; emphasizes detecting hallucinations in outputs from instruction-tuned language models (LLMs); and requires participants to predict the locations of hallucinations within the generated text. 

Mu-SHROOM focuses on identifying spans of text that represent hallucinations in LLM outputs, working within a multilingual and multi-model context. More information is available on the \href{https://helsinki-nlp.github.io/shroom/}{official task website}.

%YOU CAN MAKE IT SHORTER, LIMITED TO 150 words.
\section{Dataset Details}

The datasets provided include a sample set, validation set, and unlabeled train set, with each data point formatted as a JSON object containing a unique identifier (\texttt{id}), language (\texttt{lang}), model input question (\texttt{model\_input}), model identifier (\texttt{model\_id}), generated output (\texttt{model\_output\_text}), hard labels indicating hallucination spans, and soft labels with start, end, and empirical probability of hallucination.

The hard labels will be used to assess intersection-over-union accuracy, while soft labels will measure correlation. Participants will reconstruct soft labels during evaluation by providing the required keys for detected spans.

\subsection{Sample Dataset Illustration}
%dont use subsection here, just merge it with the superior section.

\begin{verbatim}
{"id": 1, "lang": "EN", "model_input": "When was the restoration of the Sándor Palace completed?", "model_output_text": "The restoration of Sándor Palace, also known as the Buda Castle ...", "model_id": "TheBloke/Mistral-7B-Instruct-v0.2-GGUF", "soft_labels": [{"start": 33, "prob": 0.3333333333, "end": 53}], "hard_labels": [[53, 64]]}
\end{verbatim}
%make the font smaller and allow you to present this code in about 3-4lines.

\section{Tasks}

Participants are tasked with detecting hallucinations in the provided text. 

\subsection{Evaluation Metrics}
%DONT USE SUBSECTION HERE

Participants will be evaluated based on two character-level metrics: \textbf{Intersection-over-Union} measures the overlap between characters marked as hallucinations in the gold reference and those predicted by participants, and \textbf{Probability Correlation} assesses how well the probabilities assigned by participants align with those observed by annotators.

This is a preliminary overview of our proposal; further refinements will be made as the task develops.

\end{document}
% to much blank above the title, i need to put all the content in half page!
% make the font smaller!