\documentclass{article}
\usepackage{amsmath}
\usepackage{graphicx}
\usepackage{hyperref}
\usepackage{geometry}
\usepackage{fontsize}

% Set smaller margins
\geometry{margin=1in}

% Set smaller font size
\renewcommand{\normalsize}{\fontsize{10}{12}\selectfont}

\title{SemEval-2025 Task-3 — Mu-SHROOM}
\author{}
\date{}

\begin{document}

\maketitle

\section{Introduction}

This task is part of SemEval-2025 Task-3—Mu-SHROOM, a multilingual shared-task focused on detecting hallucinations in outputs from instruction-tuned language models (LLMs). It builds on the previous SHROOM task with several key changes: it covers multiple languages, including Arabic, Chinese, English, Finnish, French, German, Hindi, Italian, Spanish, and Swedish; emphasizes LLM outputs; and requires participants to predict the locations of hallucinations within the generated text. More information is available on the \href{https://helsinki-nlp.github.io/shroom/}{official task website}.

The datasets provided include a sample set, validation set, and unlabeled train set, with each data point formatted as a JSON object containing a unique identifier (\texttt{id}), language (\texttt{lang}), model input question (\texttt{model\_input}), model identifier (\texttt{model\_id}), generated output (\texttt{model\_output\_text}), hard labels indicating hallucination spans, and soft labels with start, end, and empirical probability of hallucination. 

\small
% MAKE THIS CODE TO APPENDIX
\begin{verbatim}
{"id": 1, "lang": "EN", "model_input": "When was the restoration of the Sándor Palace completed?", 
"model_output_text": "The restoration of Sándor Palace, also known as the Buda Castle ...", 
"model_id": "TheBloke/Mistral-7B-Instruct-v0.2-GGUF", 
"soft_labels": [{"start": 33, "prob": 0.3333333333, "end": 53}], "hard_labels": [[53, 64]]}
\end{verbatim}
\normalsize

Participants are tasked with detecting hallucinations in the provided text. They will be evaluated based on two character-level metrics: \textbf{Intersection-over-Union}, which measures the overlap between characters marked as hallucinations in the gold reference and those predicted by participants, and \textbf{Probability Correlation}, which assesses how well the probabilities assigned by participants align with those observed by annotators. 

This is a preliminary overview of our proposal; further refinements will be made as the task develops.

\end{document}

% USE BOLD TEXT TO GIVE EACH TEXT A TOPIC: 1.DATASET 2.EVALUATION TOPIC NAME YOU CAN NAME THEM YOURSELF